% vim: set tw=78 tabstop=4 shiftwidth=4 aw ai:
\documentclass{beamer}

\usepackage[utf8x]{inputenc}		% diacritice
\usepackage[romanian]{babel}
\usepackage{color}			% highlight
\usepackage{alltt}			% highlight

% highlight; comment this out in case you don't input code source files
\usepackage{code/highlight}		% highlight

\usepackage{hyperref}			% folosiți \url{http://...}
					% sau \href{http://...}{Nume Link}
\usepackage{verbatim}

\mode<presentation>
{ \usetheme{Berlin} }

% Încărcăm simbolurilor Unicode românești în titlu și primele pagini
\PreloadUnicodePage{200}

\title[Super proiectul meu de licență]{Super proiectul meu de licență}
\subtitle{Sesiunea de licență -- iulie 2010}
\institute{Facultatea de Automatică și Calculatoare,\\
	Universitatea ``Politehnica'' București}
\author[Ion Ionescu]{Ion Ionescu\\
	Coordonator: Andrei Antonescu}
\date{9 iulie 2010}

\begin{document}

% Slide-urile cu mai multe părți sunt marcate cu textul (cont.)
\setbeamertemplate{frametitle continuation}[from second]

% Arătăm numărul frame-ului
%\setbeamertemplate{footline}[frame number]

\frame{\titlepage}

\frame{\tableofcontents}

% NB: Secțiunile nu sunt marcate vizual, ci doar apar în cuprins
\section{Context}

% Titlul unui frame se specifică fie în acolade, imediat după \begin{frame},
% fie folosind \frametitle
\begin{frame}{TODO}
	\begin{itemize}		% Just like normal LaTeX
		\item TODO
	\end{itemize}
\end{frame}

\section{Arhitectură și design}

\begin{frame}{TODO}
	\begin{itemize}
		\item Formulă simplă
			\begin{beamerboxesrounded}[lower=block body,shadow=true]{}
				\texttt{\#define MAX(a, b)   ((a) > (b) ? (a) : (b))}
			\end{beamerboxesrounded}
		\item TODO
	\end{itemize}
\end{frame}

\section{Implementare}

\begin{frame}{TODO}
	\begin{itemize}
		\item TODO
		\item TODO
	\end{itemize}
\end{frame}

\begin{frame}{Porțiuni de cod}
	\documentclass[so,twosetsperpage]{exam.cs.pub.ro}

\date{22 iunie 2010}
\time{Timp de lucru: 90 de minute}
\note{\textbf{Notă}: Toate răspunsurile trebuie justificate}

\gradingcount{2}{3}         % Rows and columns for the grid

\usepackage{alltt}
\usepackage{marvosym}

\newcommand{\hlstd}[1]{\textcolor[rgb]{0,0,0}{#1}}
\newcommand{\hlnum}[1]{\textcolor[rgb]{0.53,0,0.13}{#1}}
\newcommand{\hlesc}[1]{\textcolor[rgb]{1,0,1}{\bf{#1}}}
\newcommand{\hlstr}[1]{\textcolor[rgb]{0.67,0.27,0}{#1}}
\newcommand{\hldstr}[1]{\textcolor[rgb]{0,0.53,0}{#1}}
\newcommand{\hlslc}[1]{\textcolor[rgb]{0.93,0.47,0}{#1}}
\newcommand{\hlcom}[1]{\textcolor[rgb]{1,0.53,0}{#1}}
\newcommand{\hldir}[1]{\textcolor[rgb]{0,0.53,0.27}{\bf{#1}}}
\newcommand{\hlsym}[1]{\textcolor[rgb]{0,0,0}{\bf{#1}}}
\newcommand{\hlline}[1]{\textcolor[rgb]{0.2,0.2,0.2}{#1}}
\newcommand{\hlkwa}[1]{\textcolor[rgb]{0.4,0.07,0.07}{\bf{#1}}}
\newcommand{\hlkwb}[1]{\textcolor[rgb]{0,0,0.4}{\bf{#1}}}
\newcommand{\hlkwc}[1]{\textcolor[rgb]{0,0,0.4}{#1}}
\newcommand{\hlkwd}[1]{\textcolor[rgb]{0,0.27,0.4}{#1}}
\definecolor{bgcolor}{rgb}{1,1,1}   % Only necessary if you have code

\questions{questions}{6}    % Question directory and number of questions

\begin{document}
\makecontent                % Generates everything you need
\end{document}

\end{frame}
	
\section{Rezultate}

\begin{frame}{TODO}
	\begin{itemize}
		\item TODO
		\item TODO
	\end{itemize}
\end{frame}

\section{\^{I}ntrebări}

\begin{frame}{Întrebări}
  \begin{columns}
    \begin{column}[l]{0.5\textwidth}
      \begin{itemize}
        \item cuvânt cheie 1
        \item cuvânt cheie 2
        \item cuvânt cheie 3
        \item cuvânt cheie 4
        \item cuvânt cheie 5
      \end{itemize}
    \end{column}
    \begin{column}[c]{0.5\textwidth}
      \begin{figure}
        \includegraphics[scale=0.3]{img/question-mark}
      \end{figure}
    \end{column}
  \end{columns}
\end{frame}

\end{document}
